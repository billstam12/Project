\documentclass{article}
\usepackage[utf8]{inputenc}
\usepackage[greek,english]{babel}
\usepackage{lmodern}

\begin{document}
\title{\textgreek{Ανάπτυξη Λογισμικού για Αλγοριθμικά Προβλήματα:} Project 2}
\author{\textgreek{Σταματόπουλος Βασίλειος } 1115201400188}
\date{3/12/2018}
\maketitle
\section{ \textgreek{Εισαγωγή}}
\textgreek{Στην εργασία αυτή υλοποιήθηκαν 12 διαφορετικοί τρόποι} clustering,\textgreek{ όπως ακριβώς ζητήθηκε και στην εκφώνηση. \\Δηλαδή, \textbf{2 τρόποι αρχικοποίησης των δεδομένων,} τυχαία και με} k-means++ \textgreek{ 2 τρόποι }update,\textbf{ Basic Update \textgreek{με μέσα και} Partitioning Around Medoids. }\textgreek{ Επιπλεόν, το }assignment \textgreek{των σημείων γίνεται με τρεις τρόπους.} \textbf{Normal k-means, Assignment with LSH \textgreek{και} assignment with Hypercube.}
\section{ \textgreek{Αρχέια}}
\textgreek{Ο φάκελος περιέχει τα κάτωθι αρχεία στα οποία έγινε η υλοποίηση του κώδικα.}
\begin{enumerate}
\item\textbf{ cluster.c:} \textgreek{Περιέχει την} main function \textgreek{ στην οποία με τη σειρά διαβάζουμε τα }arguments. \textgreek{ύστερα διαβάζουμε το αρχείο εισόδου και στη συνέχεια αρχικοποιούμε τις μεταβλητές που θα χρειαστούμε.
Δημιουργείται ένας πίνακας }point* data, \textgreek{ο οποίος περιέχει στοιχεία} struct point  \textgreek{που έχουν την παρακάτω μορφή}
\begin{verbatim}
typedef struct point{
	long int id;
	double * coordinates;
	long long int ** g_functions;
	int centroid_id;
	int centroid2_id;
	double dist; //distance from centroids
	double dist_as_centroid;
	struct point * next;
	double silhouette;
} *point;
\end{verbatim}

\textgreek{Πάνω σε αυτά τα στοιχεία κάνουμε την διαδικασία της συσταδοποίησης.}\textgreek{Αρχικοποιούμε τα κέντρα που είναι τύπου} struct centroid \textgreek{με την ακόλουθη μορφή}
\begin{verbatim}

typedef struct centroid{
	long int id;
	int count;
	int prev_count;
	double dist;
	long long int ** g_functions;
	double * coordinates;
	double silhouette_of_cluster;
	point* assigned_points;
} *centroid;

\end{verbatim}
\textgreek{Αφού τα αρχικοποιήσουμε με την εκάστοτε μέθοδο, κάλουμε την συνάρτηση} \textbf{k-means}\textgreek{η οποία είναι αυτή που κάνει την δουλεία καλώντας στην πορεία τις απαραίτητες συναρτήσεις που βρίσκονται στο αρχείο }functions.c/functions.h \textgreek{Κατά τη διάρκεια όλης αυτής της διαδικασίας εκτυπώνονται τα κατάλληλα στοιχεία στο αρχέιο εξόδου, με τον ίδιο τρόπο που περιγράφεται και στην εκφώνηση.}

\item \textbf{functions.c:} \textgreek{Στο αρχείο αυτό βρίσκουμε όλες τις συναρτήσεις που χρησιμοποιεί ο }k-means,\textgreek{με μια μεγάλη πλειοψηφία εξ'αυτών να προέρχονται από τον φάκελο της πρώτης εργασίας. Λόγω του μεγάλου όγκου του αρχείου, οι συναρτήσεις δεν περιγράφονται εδώ, αλλά έχουν τα απαραίτητα σχόλια μέσα στο ίδιο το αρχείο.}
\item{\textbf{hashtable.c hashtable.h}} \textgreek{Τα αρχεία αυτά περιέχουν τους τύπους των στοιχείων που χρησιμοποιήθηαν καθώς και τις απαραίτητες συναρτήσεις για να λειτουργήσει ένας πίνακας κατακερματισμού, όπως οι}insert, print. \textgreek{Όπως τα δύο προηγούμενα, είναι κατάλληλα σχολιασμένα για τρίτους.}

\section{ \textgreek{Μεταγλώττιση}}
\textgreek{Το πρόγραμμα χρησιμοποιεί }makefile, \textgreek{συνεπώς μπορεί να μεταγλωττιστεί με την εντολή }make.
\section{\textgreek{Εκτέλεση}}
\textgreek{Η εκτέλεση του αρχείου ακολουθεί το πρότυπο που δώθηκε στην εκφώνηση}
\section{Version Control}
\textgreek{Για το πρόγραμμα χρησιμοποιήθηκε το} github \textgreek{για να γίνει το απαραίτητο }version control.\cite{mygit}

\section{\textgreek{Αποτελέσματα}}
\textgreek{Παρακάτω δίνονται τα αρχεία αποτελεσμάτων για τις} default \textgreek{τιμές, για τους δύο διαφορετικούς τύπους μετρικών} 
\subsection{Euclidean Distance}

\subsection{Cosine Similarity}

\subsection{\textgreek{Πορίσματα}}
\begin{thebibliography}{3}
\bibitem{mygit} 
Vasileios Stamatopoulos, Github
\\\texttt{https://github.com/billstam12/Project/tree/master/2}
\end{thebibliography}
 

\end{enumerate}
\end{document}