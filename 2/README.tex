\documentclass{article}
\usepackage[utf8]{inputenc}
\usepackage[greek,english]{babel}
\usepackage{lmodern}

\begin{document}
\title{\textgreek{Ανάπτυξη Λογισμικού για Αλγοριθμικά Προβλήματα:} Project 2}
\author{\textgreek{Σταματόπουλος Βασίλειος } 1115201400188}
\date{3/12/2018}
\maketitle
\section{ \textgreek{Εισαγωγή}}
\textgreek{Στην εργασία αυτή υλοποιήθηκαν 12 διαφορετικοί τρόποι} clustering,\textgreek{ όπως ακριβώς ζητήθηκε και στην εκφώνηση. \\Δηλαδή, \textbf{2 τρόποι αρχικοποίησης των δεδομένων,} τυχαία και με} k-means++ \textgreek{ 2 τρόποι }update,\textbf{ Basic Update \textgreek{με μέσα και} Partitioning Around Medoids. }\textgreek{ Επιπλεόν, το }assignment \textgreek{των σημείων γίνεται με τρεις τρόπους.} \textbf{Normal k-means, Assignment with LSH \textgreek{και} assignment with Hypercube.}
\section{ \textgreek{Αρχέια}}
\textgreek{Ο φάκελος περιέχει τα κάτωθι αρχεία στα οποία έγινε η υλοποίηση του κώδικα.}
\begin{enumerate}
\item\textbf{ cluster.c:} \textgreek{Περιέχει την} main function \textgreek{ στην οποία με τη σειρά διαβάζουμε τα }arguments. \textgreek{ύστερα διαβάζουμε το αρχείο εισόδου και στη συνέχεια αρχικοποιούμε τις μεταβλητές που θα χρειαστούμε.
Δημιουργείται ένας πίνακας }point* data, \textgreek{ο οποίος περιέχει στοιχεία} struct point  \textgreek{που έχουν την παρακάτω μορφή}
\begin{verbatim}
typedef struct point{
	long int id;
	double * coordinates;
	long long int ** g_functions;
	int centroid_id;
	int centroid2_id;
	double dist; //distance from centroids
	double dist_as_centroid;
	struct point * next;
	double silhouette;
} *point;
\end{verbatim}

\textgreek{Πάνω σε αυτά τα στοιχεία κάνουμε την διαδικασία της συσταδοποίησης.}\textgreek{Αρχικοποιούμε τα κέντρα που είναι τύπου} struct centroid \textgreek{με την ακόλουθη μορφή}
\begin{verbatim}

typedef struct centroid{
	long int id;
	int count;
	int prev_count;
	double dist;
	long long int ** g_functions;
	double * coordinates;
	double silhouette_of_cluster;
	point* assigned_points;
} *centroid;

\end{verbatim}
\textgreek{Αφού τα αρχικοποιήσουμε με την εκάστοτε μέθοδο, κάλουμε την συνάρτηση} \textbf{k-means}\textgreek{η οποία είναι αυτή που κάνει την δουλεία καλώντας στην πορεία τις απαραίτητες συναρτήσεις που βρίσκονται στο αρχείο }functions.c/functions.h \textgreek{Κατά τη διάρκεια όλης αυτής της διαδικασίας εκτυπώνονται τα κατάλληλα στοιχεία στο αρχέιο εξόδου, με τον ίδιο τρόπο που περιγράφεται και στην εκφώνηση.}

\item \textbf{functions.c:} \textgreek{Στο αρχείο αυτό βρίσκουμε όλες τις συναρτήσεις που χρησιμοποιεί ο }k-means,\textgreek{με μια μεγάλη πλειοψηφία εξ'αυτών να προέρχονται από τον φάκελο της πρώτης εργασίας. Λόγω του μεγάλου όγκου του αρχείου, οι συναρτήσεις δεν περιγράφονται εδώ, αλλά έχουν τα απαραίτητα σχόλια μέσα στο ίδιο το αρχείο.}
\item{\textbf{hashtable.c hashtable.h}} \textgreek{Τα αρχεία αυτά περιέχουν τους τύπους των στοιχείων που χρησιμοποιήθηαν καθώς και τις απαραίτητες συναρτήσεις για να λειτουργήσει ένας πίνακας κατακερματισμού, όπως οι}insert, print. \textgreek{Όπως τα δύο προηγούμενα, είναι κατάλληλα σχολιασμένα για τρίτους.}
\end{enumerate}
\section{ \textgreek{Μεταγλώττιση}}
\textgreek{Το πρόγραμμα χρησιμοποιεί }makefile, \textgreek{συνεπώς μπορεί να μεταγλωττιστεί με την εντολή }make.
\section{\textgreek{Εκτέλεση}}
\textgreek{Η εκτέλεση του αρχείου ακολουθεί το πρότυπο που δώθηκε στην εκφώνηση}
\section{Version Control}
\textgreek{Για το πρόγραμμα χρησιμοποιήθηκε το} github \textgreek{για να γίνει το απαραίτητο }version control.\cite{mygit}

\section{\textgreek{Αποτελέσματα}}
\textgreek{Παρακάτω δίνονται τα αρχεία αποτελεσμάτων για τις} default \textgreek{τιμές.} 
\subsection{Euclidean Distance}
Algorithm:Initialization:Random Update:Basic Assignment:k-means
CLUSTER: 0 SIZE: 4460 

CLUSTER: 1 SIZE: 47

CLUSTER: 2 SIZE: 261 

CLUSTER: 3 SIZE: 185 

CLUSTER: 4 SIZE: 46 
Clustering Time:0.702846
Shillouette of cluster 0 is: 0.000000
Shillouette of cluster 1 is: 0.991446
Shillouette of cluster 2 is: 0.946417
Shillouette of cluster 3 is: 0.970529
Shillouette of cluster 4 is: 0.981373
Algorithm:Initialization:Random Update:Basic Assignment:LSH
CLUSTER: 0 SIZE: 469 

CLUSTER: 1 SIZE: 3824 

CLUSTER: 2 SIZE: 205 

CLUSTER: 3 SIZE: 349 

CLUSTER: 4 SIZE: 152 

Clustering Time:7.101877
Shillouette of cluster 0 is: 0.000000
Shillouette of cluster 1 is: -0.848569
Shillouette of cluster 2 is: 0.960562
Shillouette of cluster 3 is: 0.815106
Shillouette of cluster 4 is: 0.919506
Algorithm:Initialization:Random Update:Basic Assignment:Hypercube
CLUSTER: 0 SIZE: 2876 

CLUSTER: 1 SIZE: 454 

CLUSTER: 2 SIZE: 435 

CLUSTER: 3 SIZE: 246 

CLUSTER: 4 SIZE: 988 

Clustering Time:1.392212
Shillouette of cluster 0 is: 0.000000
Shillouette of cluster 1 is: 0.848284
Shillouette of cluster 2 is: 0.777253
Shillouette of cluster 3 is: 0.916437
Shillouette of cluster 4 is: 0.588906
Algorithm:Initialization:Random Update:Partitioning Around Medoids Assignment:k-means
CLUSTER: 0 SIZE: 152  

CLUSTER: 1 SIZE: 202 

CLUSTER: 2 SIZE: 4220 

CLUSTER: 3 SIZE: 106 

CLUSTER: 4 SIZE: 319

Clustering Time:103.451557
Shillouette of cluster 0 is: 0.000000
Shillouette of cluster 1 is: -0.090521
Shillouette of cluster 2 is: -0.962072
Shillouette of cluster 3 is: 0.971822
Shillouette of cluster 4 is: 0.871848
Algorithm:Initialization:Random Update:Partitioning Around Medoids Assignment:LSH
CLUSTER: 0 SIZE: 231 

CLUSTER: 1 SIZE: 200 

CLUSTER: 2 SIZE: 101 

CLUSTER: 3 SIZE: 152 

CLUSTER: 4 SIZE: 4315 

Algorithm:Initialization:Random Update:Basic Assignment:Hypercube
CLUSTER: 0 SIZE: 4460  

CLUSTER: 1 SIZE: 47

CLUSTER: 2 SIZE: 261 

CLUSTER: 3 SIZE: 185 

CLUSTER: 4 SIZE: 46 
Clustering Time:0.877376
Shillouette of cluster 0 is: 0.000000
Shillouette of cluster 1 is: 0.991446
Shillouette of cluster 2 is: 0.946417
Shillouette of cluster 3 is: 0.970529
Shillouette of cluster 4 is: 0.981373
Algorithm:Initialization:Random Update:Partitioning Around Medoids Assignment:Hypercube
CLUSTER: 0 SIZE: 115 

CLUSTER: 1 SIZE: 199 

CLUSTER: 2 SIZE: 152 

CLUSTER: 3 SIZE: 4400 

CLUSTER: 4 SIZE: 133

Clustering Time:86.376880
Shillouette of cluster 0 is: 0.000000
Shillouette of cluster 1 is: -0.326268
Shillouette of cluster 2 is: -0.157874
Shillouette of cluster 3 is: -0.964690
Shillouette of cluster 4 is: 0.949213
Algorithm:Initialization:K-Means++ Update:Basic Assignment:Hypercube
CLUSTER: 0 SIZE: 940 

CLUSTER: 1 SIZE: 2972 

CLUSTER: 2 SIZE: 442 

CLUSTER: 3 SIZE: 204 

CLUSTER: 4 SIZE: 441 

Clustering Time:1.463766
Shillouette of cluster 0 is: 0.000000
Shillouette of cluster 1 is: -0.668539
Shillouette of cluster 2 is: 0.812948
Shillouette of cluster 3 is: 0.875574
Shillouette of cluster 4 is: 0.733121
Algorithm:Initialization:K-Means++ Update:Partitioning Around Medoids Assignment:Hypercube
CLUSTER: 0 SIZE: 62 

CLUSTER: 1 SIZE: 44 

CLUSTER: 2 SIZE: 28

CLUSTER: 3 SIZE: 7 

CLUSTER: 4 SIZE: 4858 

Clustering Time:129.740608
Shillouette of cluster 0 is: 0.000000
Shillouette of cluster 1 is: 0.505922
Shillouette of cluster 2 is: 0.789383
Shillouette of cluster 3 is: 0.920084
Shillouette of cluster 4 is: -0.987679


\subsection{\textgreek{Πορίσματα}}
\textgreek{Για κάποιον λόγο που δεν κατάφερα να εντοπίσω το} cluster \textgreek{ 0 πάντα είχε μηδενικό} shillouette.
\textgreek{Παρ'όλα αυτά, ο αλγόριθμος έδωσε αρκετά καλά και αναμενόμενα αποτελέσματα. Αξιοσημείωτα είναι τα εξής:}
\begin{enumerate}
\item{\textgreek{ Αρχικά παρατηρούμε πως όλοι οι αλγόριθμοι συσταδοποίησαν τα δεδομένα παρομοίως, φτίαχνωντας μια πολύ μεγάλη συστάδα και 4 μικρότερες.}}
\item{\textgreek{Σε γενικές γραμμές, οι σιλουέτες των αλγορίθμων είναι αρκετά καλές πλην μερικών περιπτώσεων. Παράξενο είναι πως το πιο απλό }version \textgreek{του αλγορίθμου έδωσε τα καλύτερα αποτελέσματα, τόσο σε χρόνο αλλά και σε σιλουέτες.}}


Tested on Linux Ubunut 18.03
\end{enumerate}
\begin{thebibliography}{3}
\bibitem{mygit} 
Vasileios Stamatopoulos, Github
\\\texttt{https://github.com/billstam12/Project/tree/master/2}
\end{thebibliography}
 
\end{document}